\section{INTRODUCCIÓN}

Las crisis alimenticias afectan a diversas partes del globo, en mayor o menor medida. Estas crisis pueden producirse por factores naturales o factores humanos como, por ejemplo, fallos en industrias, guerras, o, la que directamente se relaciona con este proyecto, agotamiento de los recursos naturales por la sobreexplotación.

Ante estas situaciones globales que pasan por revisar los valores de obtención de alimento y el cómo se gestionan los recursos, la pesca marítima está viendo como insostenible la compatibilidad de su labor con la generación de suficiente alimento vivo para suplir las necesidades exponenciales del ser humano.

El caso que más afecta a este trabajo es la acuicultura, un método alternativo que nació para obtener control sobre el desarrollo de especies muy difíciles de capturar en entornos libres. Esto se consigue a través de crianza controlada en entornos limitados, como piscifactorías o jaulas masivas en costas.

Este método de generación de alimento ha ido desarrollándose y mejorando a través de la introducción de nuevos métodos y tecnologías. Esta mejora ha permitido utilizar la acuicultura como principal método de crianza de ciertas especies, que; aun encontrándose en libertad, es más viable económicamente trabajar con ellas en estos entornos controlados. Esto permite liberar estrés de los ecosistemas marítimos.

Con estos ecosistemas tan afectados por las pescas de arrastre y, con la presión añadida por políticas limitantes en este ámbito, la pesca tradicional ha visto como la cantidad de alimento generado no crece a lo largo de los años. Esto se ve reflejado a través de los informes realizados por la FAO, siendo el último en 2022, que marcaba aún más la tendencia del impulso hacia sistemas basados en acuicultura, observable en la Figura 1.
