\section*{RESUMEN}
\thispagestyle{abstract}
En los últimos años la visión por ordenador ha tomado un papel muy importante en la vida cotidiana. Dentro de este campo encontramos técnicas como el análisis de 
flujo óptico o las redes neuronales de tipo \texttt{CNN}, siendo \texttt{YOLO} una arquitectura muy popular.

Con estas nuevas técnicas se busca automatizar procesos en la industria de la acuicultura, tanto de producción como experimentales. En el estado del arte se presentan 
usos actuales recientes en piscifactorías para controlar bancos de peces, detectar especímenes y contabilización de los mismos. Algunas de las ventajas del uso de estas 
líneas de investigación es la mejora de las condiciones de los peces en la acuicultura. 

En este trabajo se trata la automatización del experimento \textit{NetTest}, que se utiliza para analizar la cantidad de estrés que genera en los animales el cambio de 
condiciones en los entornos en los que viven. Este experimento se basa en contar el número de movimientos realizados por un pez en una red fuera del agua, y, actualmente 
es realizado de forma manual.

Para conseguir esto, se ha desarrollado una aplicación de detección automática y contabilización del número de movimientos realizados por truchas en videos del experimento 
\textit{NetTest}. Con este objetivo, se ha formado un conjunto de datos y se ha entrenado un modelo de \texttt{YOLOv8}, el cual procesa el video y devuelve los resultados de 
las \texttt{Bounding Boxes} asociadas a cada trucha.

Una vez localizada la trucha se ha diseñado un algoritmo que utiliza el área de la \texttt{bounding box} y elementos como el desplazamiento del centroide para marcar en qué 
fotograma del video se ha producido un movimiento de escape. Finalmente, para poder facilitar el uso de esta herramienta, se ha desarrollado una interfaz gráfica para 
el análisis de los resultados en los experimentos \textit{NetTest}.

Para implementar todas las funcionalidades se han utilizado herramientas como la ejecución paralela y la inferencia de forma transparente al \texttt{HardWare} a través del uso 
de múltiples formatos de modelo de red neuronal.

Como resultados, el modelo \texttt{YOLOv8n} se ha entrenado hasta reducir el error asociado a la \texttt{Bounding Box} y maximizar la puntuación \texttt{F1}. Finalmente, la 
comparación de resultados con una serie de datos etiquetados en 64 videos ha demostrado una tasa de error de \texttt{$ \pm 20\% $}. Estos mismos resultados que antes se tardaban 
en obtener semanas, con esta herramienta se han podido obtener en menos de una hora.