\section*{RESUMEN}
\thispagestyle{abstract}
En los últimos años la visión por ordenador ha tomado un papel muy importante en la vida cotidiana. Dentro de este campo encontramos técnicas como el análisis de 
flujo óptico o las redes neuronales de tipo \texttt{CNN}, siendo \texttt{YOLO} una arquitectura muy popular.

En este trabajo se ha desarollado una aplicación de detección y contabilización del número de movimientos realizados por truchas en videos del experimento \textit{NetTest}. 
Para esto se ha formado un conjunto de datos y se ha entrenado un modelo de \texttt{YOLOv8}, el cual procesa el video y devuelve los resultados de las \texttt{Bounding Boxes} 
asociadas a cada trucha.\newline Posteriormente los datos son transformados y se les aplica un algoritmo para estimar los fotogramas en los que ha sucedido un movimiento. 
Encima de esto, se ha desarrollado una interfaz gráfica que realiza todo esto de forma transparente a los conocimientos del usuario. Para implementar todas las funcionalidades 
se han utilizado herramientas como la ejecución paralela y la inferencia de forma transparente al \texttt{HardWare}.

Como resultados, el modelo \texttt{YOLOv8n} se ha entrenado hasta reducir el error asociado a la \texttt{Bounding Box} y maximizar la puntuación \texttt{F1}. Finalmente, la 
comparación de resultados con una serie de datos etiquetados en 64 videos ha demostrado una tasa de error de \texttt{$ \pm 20\% $}.