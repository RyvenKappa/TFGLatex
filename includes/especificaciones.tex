\section{REQUISITOS Y RESTRICCIONES DE DISEÑO}

Como proyecto de ingeniería, el análisis previo de requisitos y restricciones del sistema se ha realizado para modelar el comportamiento esperado 
resultante del desarrollo de la solución propuesta

Dentro del análisis de requisitos, se han diferenciado entre funcionales(definiciones del comportamiento del sistema) y no funcionales(definiciones de atributos de control de calidad).

\subsection{Análisis de requisitos}
\begin{table}[H]
    \begin{center}
        \begin{tabular}{p{0.1\textwidth} | p{0.7\textwidth} p{0.1\textwidth}}
            Req. ID & Descripción de requisito funcional & Prioridad\\
            \hline
            0& Los cálculos automáticos deben ser transparentes a los conocimientos del usuario & Alta\\
            \hline
            1& El usuario deberá poder seleccionar el nivel de agresividad del sistema & Media\\
            \hline
            2& Se deberá llevar un registro por usuario de las predicciones anteriores & Baja\\
            \hline
            3& El usuario debe poder cambiar de usuario & Baja\\
            \hline
            3& El usuario debe poder ver cuáles usuarios existen & Baja\\
            \hline
            4& La inferencia debe hacerse posterior a la selección y confirmación del video & Alta\\
            \hline
            5& El usuario deberá poder seleccionar diferentes formatos de video sobre los que inferir de forma transparente & Alta\\
            \hline
            6& Si ocurre un error en la selección de fichero, se informará por un pop-up & Alta\\
            \hline
            7& La inferencia se realizará de forma transparente al usuario en una pantalla de carga & Alta\\
            \hline
            8& La inferencia se podrá cancelar para evitar perdidas de tiempo & Media\\
            \hline
            9& En la pantalla de carga se informará de cuanto queda para acabar la inferencia & Alta\\
            \hline
            10& Si ocurre un error en la inferencia se deberá informar con un pop-up y volver a la pantalla inicial & Alta\\
            \hline
            11& Se deberá hacer la inferencia en paralelo para no colgar la aplicación & Alta\\
            \hline
            12& Si se completa la inferencia, deberá pasarse a la pantalla de resultados & Alta\\
            \hline
            13& En la pantalla de resultados el usuario deberá poder elegir guardarlos en csv y excel & Alta\\
            \hline
            14& En la pantalla de resultados el usuario debe poder verificar los resultados obtenidos & Alta\\
            \hline
            15& En la pantalla de resultados el usuario debe poder añadir o eliminar movimientos de los resultados obtenidos & Media\\
            \hline
            16& En la pantalla de resultados el usuario debe poder volver a la pantalla inicial & Media\\
            \hline
            17& En la pantalla de resultados el usuario puede elegir guardar estos resultados en su perfil (sin guardar el video, solo el registro de movimientos y el nombre del video) & Bajo\\
            \hline
            18& Si el usuario guarda un resultado por primera vez, ese nuevo usuario se quedará registrado en la aplicación  & Bajo\\
            \hline
        \end{tabular} 
    \end{center}
    \caption{Análisis de requisitos funcionales}
    \label{ReqFuncionales}
\end{table}

\begin{table}[H]
    \begin{center}
        \begin{tabular}{p{0.1\textwidth} | p{0.7\textwidth} p{0.1\textwidth}}
            Req. ID & Descripción de requisito no funcional & Prioridad\\
            \hline
            0& Descripción 0\newline prueba de meter mas texto en mas lineas & Alta\\
            \hline
            1& Descripción 1 & Alta\\
            \hline
            2& Descripción 2 & Alta\\
            \hline
            3& Descripción 3 & Alta\\
            \hline
            4& Descripción 4 & Alta\\
            \hline
            5& Descripción 5 & Alta\\
            \hline
            6& Descripción 6 & Alta\\
            \hline
        \end{tabular} 
    \end{center}
    \caption{Análisis de requisitos no funcionales}
    \label{ReqNoFuncionales}
\end{table}

\clearpage

\subsection{Análisis de restricciones}

\begin{table}[H]
    \begin{center}
        \begin{tabular}{p{0.1\textwidth} | p{0.8\textwidth}}
            Rest. ID & Descripción de restricción\\
            \hline
            0& Descripción 0\newline prueba de meter mas texto en mas lineas\\
            \hline
            1& Descripción 1\\
            \hline
            2& Descripción 2\\
            \hline
            3& Descripción 3\\
            \hline
            4& Descripción 4\\
            \hline
            5& Descripción 5\\
            \hline
            6& Descripción 6\\
            \hline
        \end{tabular} 
    \end{center}
    \caption{Análisis de restricciones}
    \label{Restricciones}
\end{table}