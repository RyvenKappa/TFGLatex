\section{REQUISITOS Y RESTRICCIONES DE DISEÑO}

Como proyecto de ingeniería, el análisis previo de requisitos y restricciones del sistema se ha realizado para modelar el comportamiento esperado 
resultante del desarrollo de la solución propuesta

Dentro del análisis de requisitos, se han diferenciado entre funcionales(definiciones del comportamiento del sistema) y no funcionales(definiciones de atributos de control de calidad).

\subsection{Análisis de requisitos}
\begin{table}[H]
    \begin{center}
        \begin{tabular}{p{0.1\textwidth} | p{0.7\textwidth} p{0.1\textwidth}}
            Req. ID & Descripción de requisito funcional & Prioridad\\
            \hline
            0& Los cálculos automáticos deben ser transparentes a los conocimientos de los usuarios & Alta\\
            \hline
            1& El usuario deberá poder seleccionar el nivel de agresividad del sistema & Media\\
            \hline
            2& Se deberá llevar un registro por usuario de las predicciones anteriores & Baja\\
            \hline
            3& El usuario debe poder cambiar de usuario & Baja\\
            \hline
            3& El usuario debe poder ver cuáles usuarios existen & Baja\\
            \hline
            4& La inferencia debe hacerse posterior a la selección y confirmación del video & Alta\\
            \hline
            5& El usuario deberá poder seleccionar diferentes formatos de video sobre los que inferir de forma transparente & Alta\\
            \hline
            6& Si ocurre un error en la selección de fichero, se informará por un pop-up & Alta\\
            \hline
            7& La inferencia se realizará de forma transparente al usuario en una pantalla de carga & Alta\\
            \hline
            8& La inferencia se podrá cancelar para evitar perdidas de tiempo & Media\\
            \hline
            9& En la pantalla de carga se informará de cuanto queda para acabar la inferencia & Alta\\
            \hline
            10& Si ocurre un error en la inferencia se deberá informar con un pop-up y volver a la pantalla inicial & Alta\\
            \hline
            11& Se deberá hacer la inferencia en paralelo para no colgar la aplicación & Alta\\
            \hline
            12& Si se completa la inferencia, deberá pasarse a la pantalla de resultados & Alta\\
            \hline
            13& En la pantalla de resultados el usuario deberá poder elegir guardarlos en csv y excel & Alta\\
            \hline
            14& En la pantalla de resultados el usuario debe poder verificar los resultados obtenidos & Alta\\
            \hline
            15& En la pantalla de resultados el usuario debe poder añadir o eliminar movimientos de los resultados obtenidos & Media\\
            \hline
            16& En la pantalla de resultados el usuario debe poder volver a la pantalla inicial & Media\\
            \hline
            17& En la pantalla de resultados el usuario puede elegir guardar estos resultados en su perfil (sin guardar el video, solo el registro de movimientos y el nombre del video) & Baja\\
            \hline
            18& Si el usuario guarda un resultado por primera vez, ese nuevo usuario se quedará registrado en la aplicación  & Baja\\
            \hline
            19& La aplicación debe tener incorporado en el menu un botón para abrir un manual  & Media\\
            \hline
        \end{tabular} 
    \end{center}
    \caption{Análisis de requisitos funcionales}
    \label{ReqFuncionales}
\end{table}

\begin{table}[H]
    \begin{center}
        \begin{tabular}{p{0.1\textwidth} | p{0.7\textwidth} p{0.1\textwidth}}
            Req. ID & Descripción de requisito no funcional & Prioridad\\
            \hline
            20& El sistema automático deberá permitir a los científicos realizar su tarea en menos tiempo que manualmente & Alta\\
            \hline
            21& El sistema debe poder ser manejado por cualquier usuario que sepa realizar tareas de ofimática & Alta\\
            \hline
            22& El sistema debe poder compilarse en un ejecutable para que el usuario no necesite manejar dependencias  & Alta\\
            \hline
            23& Los errores no deben bloquear la aplicación, deben ser manejados y controlados correctamente en lo posible & Media\\
            \hline
            24& Las entradas de datos si fuesen necesarias se harán de forma simple para el usuario, ya sea a través de darle a botones o usar el ratón & Media\\
            \hline
            25& El sistema debe aprovechar al máximo los recursos \texttt{HardWare} a ser posible, seleccionando el mejor modelo para el ordenador en el que se lanza la aplicación & Alta\\
            \hline
            26& La comunicación entre procesos debe manejar problemas de concurrencia & Alta\\
            \hline
            27& La inferencia se realizará fotograma por fotograma para limitar el uso de memoria RAM del sistema al mínimo necesario & Alta\\
            \hline
            28& Se utilizará una librería para la interfaz de usuario que permita el renderizado sobre OpenGL o similares para descargar la \texttt{\acrshort{cpu}} & Alta\\
            \hline
            29& La capa de inferencia debe dar unos resultados transparentes independientemente del modelo utilizado & Baja\\
            \hline
            30& El modelo debe poder cambiar en tamaño para ajustarse a diferentes requerimientos en un futuro & Media\\
            \hline
            31& El fichero de datos de usuario debe poder ser portátil para que el ejecutable lo detecte y cargue los datos de usuario si existen & Baja\\
            \hline
            32& El usuario no tiene por qué especificar el número de truchas que contiene el video, se estima solo & Media\\
            \hline
        \end{tabular} 
    \end{center}
    \caption{Análisis de requisitos no funcionales}
    \label{ReqNoFuncionales}
\end{table}

\subsection{Análisis de restricciones}

\begin{table}[H]
    \begin{center}
        \begin{tabular}{p{0.1\textwidth} | p{0.8\textwidth}}
            Rest. ID & Descripción de restricción\\
            \hline
            0& La aplicación debe poder correr en un ordenador con \texttt{Windows}\\
            \hline
            1& La aplicación se implementará usando el lenguaje de programación \texttt{Python}\\
            \hline
            2& Se usará en la medida de lo posible, librerías y herramientas de código abierto y/o gratis\\
            \hline
            3& La aplicación se desarrollará usando \texttt{Git} para realizar un control de versiones\\
            \hline
            4& Los datos de la aplicación que aporten persistencia se guardarán en claro, no tienen un riesgo de seguridad\\%TODO pedro truco
            \hline
            5& Los videos de entrada no podrán contener menos de 50 fotogramas\\
            \hline
        \end{tabular} 
    \end{center}
    \caption{Análisis de restricciones}
    \label{Restricciones}
\end{table}