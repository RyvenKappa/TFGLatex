\section{REQUISITOS Y RESTRICCIONES DE DISEÑO}

Como proyecto de ingeniería, el análisis previo de requisitos y restricciones del sistema se ha realizado para modelar el comportamiento esperado 
resultante del desarrollo de la solución propuesta. \newline
Dentro del análisis de requisitos, se han diferenciado entre funcionales(definiciones del alto nivel) y no funcionales(definiciones técnicas y de bajo nivel).

\subsection{Análisis de requisitos}
\begin{table}[H]
    \begin{center}
        \begin{tabular}{p{0.1\textwidth} | p{0.7\textwidth} p{0.1\textwidth}}
            Req. ID & Descripción de requisito no funcional & Prioridad\\
            \hline
            0& Descripción 0\newline prueba de meter mas texto en mas lineas & Alta\\
            \hline
            1& Descripción 1 & Alta\\
            \hline
            2& Descripción 2 & Alta\\
            \hline
            3& Descripción 3 & Alta\\
            \hline
            4& Descripción 4 & Alta\\
            \hline
            5& Descripción 5 & Alta\\
            \hline
            6& Descripción 6 & Alta\\
            \hline
        \end{tabular} 
    \end{center}
    \caption{Análisis de requisitos funcionales}
    \label{ReqFuncionales}
\end{table}

\begin{table}[H]
    \begin{center}
        \begin{tabular}{p{0.1\textwidth} | p{0.7\textwidth} p{0.1\textwidth}}
            Req. ID & Descripción de requisito no funcional & Prioridad\\
            \hline
            0& Descripción 0\newline prueba de meter mas texto en mas lineas & Alta\\
            \hline
            1& Descripción 1 & Alta\\
            \hline
            2& Descripción 2 & Alta\\
            \hline
            3& Descripción 3 & Alta\\
            \hline
            4& Descripción 4 & Alta\\
            \hline
            5& Descripción 5 & Alta\\
            \hline
            6& Descripción 6 & Alta\\
            \hline
        \end{tabular} 
    \end{center}
    \caption{Análisis de requisitos no funcionales}
    \label{ReqNoFuncionales}
\end{table}

\clearpage

\subsection{Análisis de restricciones}

\begin{table}[H]
    \begin{center}
        \begin{tabular}{p{0.1\textwidth} | p{0.8\textwidth}}
            Rest. ID & Descripción de restricción\\
            \hline
            0& Descripción 0\newline prueba de meter mas texto en mas lineas\\
            \hline
            1& Descripción 1\\
            \hline
            2& Descripción 2\\
            \hline
            3& Descripción 3\\
            \hline
            4& Descripción 4\\
            \hline
            5& Descripción 5\\
            \hline
            6& Descripción 6\\
            \hline
        \end{tabular} 
    \end{center}
    \caption{Análisis de restricciones}
    \label{Restricciones}
\end{table}