\subsection{Desarrollo de aplicación completa}

El siguiente objetivo del trabajo fue desarrollar una aplicación completa que permitiese a los usuarios obtener los resultados de forma fácil. Esto fue decidido principalmente por dos razones:

\begin{itemize}
    \item Era necesario cumplir el requisito que afecta al nivel de conocimiento necesario para manejar la herramienta. Con una aplicación que tuviese interfaz gráfica, se podría manejar mucho mejor que 
    obligar al usuario a usar una consola de comandos o a conocer como funcionar las diferentes dependencias de las librerías usadas.
    \item Servía como oportunidad para mejorar los servicios que era capaz de proporcionar el sistema, además de poder diseñar la aplicación para poder aprovechar de forma eficiente los recursos del 
    \texttt{HardWare}.
\end{itemize}

Para desarrollar esta aplicación primero se seleccionó la librería de \acrfullr{gui}. Este paso fue importante porque existen muchas posibles librerías en el entorno de \texttt{Python}, pero cada una 
aporta diferentes utilidades y abstracciones, siendo las más comunes:
\begin{itemize}
    \item \texttt{TKinter}: es el \texttt{framework} más usado en \texttt{Python}. Es una librería simple y que no se suele usar para manejar datos multimedia, ya que es muy simple.
    \item \texttt{PyQT}: librería que internamente usa el \texttt{framework} \texttt{QT}. Es mucho más potente y capaz de manejar videos y elementos complejos.
    \item \texttt{Kivy}: librería pensada para el desarrollo de aplicaciones móviles.
    \item \texttt{PySimpleGUI}: es una librería \texttt{wrapper} sobre \texttt{Tkinter} y similares, por lo tanto es muy simple, pero limita mucho.
    \item \texttt{Remi}: se usa para crear interfaces web.
    \item \texttt{DearPyGUI}\cite{HoffstadtDearPyGuiDear}: librería muy reciente que se centra en dar al usuario la máxima eficiencia posible, usando aceleración \texttt{HardWare} a través de \texttt{OpenGL}.
\end{itemize}

\vspace{2\baselineskip}

Al estar trabajando con redes neuronales, queremos que la interfaz gráfica utilice la menor cantidad de recursos posible y de la manera más eficiente. Además, como se verá en los siguientes puntos, 
se implementaron partes multimedia en la aplicación, por lo tanto fue necesaria una librería centrada en la eficiencia, lo cual dejaba como posibles opciones 
\texttt{PyQT} y \texttt{DearPyGUI}, sin embargo por motivos personales del autor, la librería seleccionada fue \texttt{DearPyGUI}.

La selección se debe a la experiencia del autor en otros trabajos con la misma librería para realizar aplicaciones de manejo de cámaras IP. Al conocer ya la librería, se reducirían 
los posibles problemas y la necesidad de mirar documentación. Además de esta experiencia pasada, quedó patente que \texttt{DearPyGUI} es capaz de manejar los recursos del ordenador de forma muy eficiente.

\clearpage
\subsubsection{Diseño de interfaz para la aplicación}

Para realizar el diseño de la aplicación, se diseñó a través de la herramienta \texttt{DrawIO}, un esquema general del flujo y ventanas de la aplicación (completo en el \hyperref[esquema:FlujoVentanas]{anexo c}).\newline
Usando este esquema como idea general, se fue desarrollando la aplicación poco a poco, desde la ventana inicial hasta la ventana de datos

\begin{itemize}
    \item Ventana inicial de la interfaz
\end{itemize}