\section*{ABSTRACT}
\thispagestyle{abstract}
In recent years, the development of computer vision tools has become very important in our lives. In this field there are techniques such as optical flow analysis or 
convolutional neural networks, being YOLO one of the most popular architectures.

In this document, an application for detecting and counting trout movements in \textit{NetTest} videos is presented. To accomplish this, a data set was made and 
a \texttt{YOLOv8} model was trained. The model task is to process the video and return data related to the \texttt{Bounding Boxes} detected for each trout.\newline
Later, all this data is transformed, and an algorithm is applied to estimate the frame and the number of movements. In top of this, a graphical user interface has been 
developed to do this tasks in a transparent way to the knowledge of the user. To implement all these functionalities, different tools were used such as multiprocessing 
and \texttt{HardWare} independent inference.

In the validation of the results, the \texttt{YOLOv8n} model was trained till the \texttt{Box Loss} error was minimized and the \texttt{F1} metric maximized. Finally, 
the movement results were compared to labels obtain from scientist of the \textit{NetTest}. The labels and the results were compared in 64 videos, obtaining an error 
of \texttt{$ \pm 20\% $}.