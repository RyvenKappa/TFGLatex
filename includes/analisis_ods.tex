\section{ANÁLISIS DE IMPACTO DEL PROYECTO}

Como se discutió en la introducción de este trabajo, el cambio climático cada vez está tomando más importancia en la sociedad. Dentro de todas las herramientas que existen para reducir su 
efecto, está la investigación para la reducción de huella de carbono en la generación de alimento. Especialmente, se están realizando muchas investigaciones con la acuicultura, ya que es un 
método muy seguro y controlado para cubrir las necesidades de proteína de origen marítimo y evitar la pérdida de los ecosistemas marítimos.

Este trabajo tiene un impacto social, siendo este no sobre el ser humano, sino sobre los animales. La herramienta que se ha generado permite a investigadores parametrizar de forma experimental 
las consecuencias que tiene un tratamiento u otro. Esto tiene como objetivo determinar técnicas que aumenten el bienestar animal en las piscifactorías.

En el análisis de impacto económico, este trabajo mejora sistemas que antes tomarían demasiado tiempo y serían propensos a errores humanos, lo cual se transforma en una reducción del dinero 
perdido en analizar recursos como videos.

Dentro del análisis de impacto también aparece la eliminación de la brecha digital a la hora de gestionar resultados de experimentos con áreas que no son la ingeniería.

\subsubsection*{Objetivos de Desarrollo Sostenible}

Este trabajo busca alinearse con los \acrshort{ods} definidos por la Unión Europea. Dentro de los objetivos individuales que se ven afectados por este trabajo tenemos:
\begin{itemize}
    \item \textbf{Trabajo decente y crecimiento económico}: los investigadores que pierden más tiempo analizando videos que realizando experimentos y obteniendo conclusiones van a ver esta herramienta 
    para automatizar procesos llenos de errores humanos.
    \item \textbf{Industria, innovación e infraestructura}: este trabajo busca dar herramientas para mejorar las condiciones animales en la industria de la acuicultura.
    \item \textbf{Producción y consumo responsables}: si se mejora la calidad animal en la acuicultura, la producción aumentará y será menos necesario utilizar los entornos marítimos para realizar pesca 
    masiva.
    \item \textbf{Acción por el clima}: mejorar las condiciones de producción en la acuicultura reduce los beneficios que produce la pesca de alta mar.
    \item \textbf{Vida submarina}: la obtención de proteína a través de la acuicultura consigue aliviar los entornos marinos.
\end{itemize}