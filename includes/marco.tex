\section{MARCO TECNOLÓGICO}

El problema que se busca solucionar en este trabajo tiene relación directa con el campo de estudio de la visión por ordenador. Este campo de la Ingeniería Informática busca, 
a través de sistemas de captación y procesamiento de imágenes, generar información y automatizar procesos con ordenadores. \newline Dentro de este campo se han desarrollado tecnologías que usamos día a día
\cite{szeliskiComputerVisionAlgorithms2022}:

\begin{itemize}
    \item \textbf{\textit{\acrfullr{ocr}}} para realizar reconocimiento de texto en imágenes y documentos.
    \item \textbf{Detección de fallos en maquinaria} en procesos de fabricación.
    \item \textbf{Fotogrametría} para mapear imágenes a modelos 3D.
    \item \textbf{Imagen médica} para uso en operaciones a tiempo real\cite{NEMESIS3DCM}.
    \item \textbf{Detección de objetos} en imágenes o videos.
    \item \textbf{Detección de movimiento} y seguimiento de objetos en escenas.
\end{itemize}

Las aplicaciones que usan herramientas basadas en este tipo de soluciones suelen seguir una estructura similar a la siguiente:

\begin{enumerate}
    \item Capturar información a través de cámaras o flujos de video IP de diferentes características.
    \item (A veces) Realizar un preprocesamiento de las imágenes para adaptarlas al sistema.
    \item Procesar las imágenes para extraer información relevante para el sistema.
    \item Transformar la información extraída a través de algoritmos inteligentes para construir una base de datos que sea útil para el usuario de la aplicación.
\end{enumerate}

En el caso de este trabajo, se tratan problemas de detección de objetos y seguimiento de movimientos. Este tipo de tareas se han llevado a cabo antiguamente a través de algoritmos complejos y deterministas, pero en los últimos 
años, con el auge del \textit{Machine Learning}, se ha creado un nuevo paradigma.

En los siguientes puntos se va a describir de forma breve las técnicas basadas en 

\subsection{Aprendizaje automático y Deep Learning}



\subsection{Detección de Objetos}

La detección de objetos como tarea surge de la necesidad de analizar la existencia de objetos en imágenes y; opcionalmente, posicionarlos en las mismas.Para realizar este tipo de tareas se han ido 
desarrollando técnicas tradicionales basadas en algoritmos extremadamente deterministas, pero, con el auge del \textit{Machine Learning}, a partir de 2012\cite{zouObjectDetection202023} 
el paradigma pasó a centrarse en la creación de sistemas detectores basados en \textit{Deep Learning} (Sistemas de Redes Neuronales con una gran cantidad de capas ocultas).

Ambas técnicas se basan en la extracción de características de las imágenes, esto es, la caracterización de imágenes o zonas de interés a través de algoritmos o filtros. Ejemplos de características 
muy usadas que un humano puede entender son:\begin{itemize}
    \item \textbf{Nitidez} de la imagen aplicando filtros laplacianos como el visto en la \autoref{fig:LaplaceKernel}.
    Con esto nos referimos a x

    hola prueba
    \item Histogramas de color.
    \item Existencia de formas específicas en la imagen.
\end{itemize}

\begin{figure}[H]
    \centering
    \(
    \begin{vmatrix}
        0 & -1 & 0 \\
        -1 & 4 & -1 \\
        0 & -1 & 0
    \end{vmatrix}
    \)
    \caption{Aspecto de un \textit{kernel} de laplace de tamaño \texttt{3x3}}
    \label{fig:LaplaceKernel}
\end{figure}


\subsection{Seguimiento de Movimientos}

\subsection{Entornos de desarrollo, sistemas de despliegue y \textit{Hardware} transparente}