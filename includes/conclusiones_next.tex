\section{CONCLUSIONES Y TRABAJOS FUTUROS}

Los experimentos actuales en la ciencia cada vez manejan mayores cantidades de datos. La capacidad de automatizar la obtención de los datos, su procesado y su análisis es uno de los campos más importantes 
de la tecnología moderna. Dentro de esto encontramos la visión por ordenador como campo que engloba diversas tecnologías para procesar imagen.

Dentro la visión por ordenador, uno de los paradigmas más eficientes en la actualidad para conseguir analizar videos son las arquitecturas de redes neuronales \texttt{CNN}l que aprovechan un planteamiento de 
operaciones por matrices para poder manejar grandes cantidades de datos a la vez. Una de las \texttt{CNN} más populares actualmente es \texttt{YOLOv8} por su rendimiento y eficiencia.

A través de pruebas comparativas, se ha observado que respecto a métodos tradicionales como el análisis de flujo óptico, las arquitecturas \texttt{CNN}; más específicamente \texttt{YOLO}, se pueden usar 
para caracterizar el movimiento de truchas de forma extremadamente rápida. Estos sistemas son capaces de reconocer correctamente truchas en movimiento y con una tasa de error muy baja con poco tiempo de entrenamiento.

En este trabajo se ha usado el modelo \texttt{YOLO} entrenado para la detección de truchas para construir una aplicación que cuente el número de movimientos. Para conseguir esto se han aprovechado técnicas como 
los sistemas basados en ejecución paralela y la correcta elección del modelo dependiendo del \texttt{HardWWare} disponible del ordenador. Con esto se han conseguido obtener resultados como el 
procesamiento de 64 videos en 1 hora, obteniendo una tasa de error respecto a los movimientos detectados sobre movimientos previamente etiquetados en su mayoría de entre -20\% y 20\%. \newline
Estos resultados son muy buenos y cumplen con los objetivos marcados en el proyecto.

Con todo esto se ha conseguido construir una aplicación capaz de funcionar a tiempo real, que no necesita conocimientos de tecnologías sobre las redes neuronales, lenguajes de programación o 
uso de una consola de comandos para ser usada. Esto permite potenciar la automatización de procesos experimentales en campos donde el uso de este tipo de herramientas va retrasado.

A través de este trabajo, se ha podido estudiar el desarrollo de soluciones aplicando técnicas de ingeniería (Planteamiento → Prueba de concepto → Desarrollo → Validación de resultados → Entrega final), entendiendo 
los requisitos que puede tener una persona ajena al mundo de la ingeniería y sabiendo definir las limitaciones que puede tener la puesta a prueba de un concepto en un entorno real donde hay variables que no se 
pueden controlar.

\vspace{2\baselineskip}

Las futuras líneas que se plantean en este trabajo son las siguientes:
\begin{itemize}
    \item Mejora de la interfaz de usuario: principalmente añadir funcionalidades centradas en la accesibilidad.
    \item Mejora de la red neuronal: hacer un \texttt{fine-tunning} de hiperparámetros y realizar un entrenamiento lo mejor posible teniendo un conjunto de datos mucho más amplio. Esto permitiría ajustar completamente la 
    \texttt{Bounding Box} de la trucha. También sería interesante entrenar la red neuronal para trabajar con peces similares.
    \item Despliegue sobre dispositivos IoT y creación de un cliente ligero: esto permitiría distribuir el sistema y que un usuario pudiese usarlo sin tener un ordenador de altas prestaciones.
    \item Mejorar el algoritmo de control de movimiento: plantear una alternativa mucho más flexible e inteligente, como un árbol de decisión.
\end{itemize}