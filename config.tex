
%Sección de paquetes
\usepackage[utf8]{inputenc}
\usepackage[style=ieee]{biblatex}
\usepackage{csquotes}
\usepackage[top=2.5cm,bottom=2.5cm,left=4cm,right=2cm]{geometry}
\usepackage[spanish,es-tabla]{babel}
\usepackage{float}
\usepackage{caption}
\usepackage{pdfpages} % Para insertar la portada en formato PDF.
\usepackage[hidelinks]{hyperref} % Para urls.
\usepackage{longtable} % Para tablas largas.
\usepackage{graphicx} % Para cargar imagenes
\usepackage{titlesec}
\usepackage{fancyhdr}
\usepackage[parfill]{parskip}
\usepackage[acronym,nogroupskip]{glossaries}
\usepackage{todonotes}
\usepackage{dirtree}
\usepackage{subcaption}
\usepackage{mathtools}
\usepackage{amsmath}
\raggedbottom

\addbibresource{TFGDiego.bib}

%Eliminar la sangría y otros ajustes de los headers
\setlength{\parindent}{0px}
\setlength{\headheight}{13.07225pt}
%Glosarios
\makenoidxglossaries

%ACRONIMOS

\newacronym{fao}{FAO}{Food and Agriculture Organization of the United Nations}
\newacronym{i+d}{I+D}{Investigación y Desarrollo}
\newacronym{ETSIAAB}{ETSIAAB}{Escuela Técnica Superior de Ingeniería Agronómica, Alimentaria y de Biosistemas}
\newacronym{pran}{PRAN}{Grupo de Investigación de Producción Animal}
\newacronym{mts}{MTS}{MPEG Transport Stream}
\newacronym{fps}{FPS}{Frames Per Second}
\newacronym{mov}{MOV}{QuickTime File Format}
\newacronym{ltr}{LTR}{Left To Right}
\newacronym{ocr}{OCR}{Optical Character Recognition}
\newacronym{mlp}{MLP}{MultiLayer Perceptron}
\newacronym{roi}{ROI}{Region Of Interest}
\newacronym{cnn}{CNN}{Convolutional Neural Networks}
\newacronym{map}{mAP}{Mean Average Precision}
\newacronym{citsem}{CITSEM}{Centro de Investigación en Tecnologías Software y Sistemas Multimedia para la Sostenibilidad}
\newacronym{gpu}{GPU}{Graphics Processing Unit}
\newacronym{cpu}{CPU}{Central Processing Unit}
%GLOSARIO Y DEFINICIONES


\renewcommand*\glspostdescription{\hfill}
\newcommand\acrfullr[2][]{\acrshort[#1]{#2} (\acrlong[#1]{#2})}


%Nuevos estilos de páginas por errores
\fancypagestyle{specialpage}{%
  \fancyhf{}
  \fancyhead[L]{1. LISTA DE ACRÓNIMOS}
  \fancyfoot[EL]{\thepage}
  \fancyfoot[OR]{\thepage}
}

\fancypagestyle{abstract}{
    \fancyhf{}
    \renewcommand{\headrulewidth}{0pt} % Asegurarse de que la barra negra también se elimine en esta página
    \fancyfoot[EL]{\thepage}
    \fancyfoot[OR]{\thepage}
}